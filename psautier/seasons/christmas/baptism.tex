\begin{center}\normalsize THE BAPTISM OF THE LORD\\
\footnotesize FEAST
\end{center}

\begin{flushleft}\normalsize{\uppercase{FIRST VESPERS\\}}\end{flushleft}

\noindent\small{\uppercase{Hymn}}\normalsize
\begin{verse}
When Jesus comes to be baptized,\\
He leaves the hidden years behind,\\
the years of safety and of peace,\\
to bear the sins of all mankind.\\!

The Spirit of the Lord comes down,\\
anoints the Christ to suffering,\\
to preach the word, to free the bound,\\
and to the mourner, comfort bring.\\!

He will not quench the dying flame,\\
and what is bruised he will not break,\\
but heal the wound injustice dealt,\\
and out of death his triumph make.\\!

Our everlasting Father, praise,\\
with Christ, his well-beloved Son,\\
who with the Spirit reigns serene,\\
untroubled Trinity in One. Amen.\\!
\end{verse}
\begin{flushright}\tiny -Hamburg, 1690\end{flushright}

\noindent\small{\uppercase{PSALMODY}\\}
\textsc{of First Vespers of the Epiphany, p. \pageref{christmas:epiphany:firstVespers:psalmody}\\}

\noindent\small{\uppercase{READING:}}   Acts 10:37-38 \textbf{  What has happened all over Judea, beginning in Galilee after the baptism that John preached, how God anointed Jesus of Nazareth with the holy Spirit and power. He went about doing good and healing all those oppressed by the devil, for God was with him.\\}

\noindent\small{\uppercase{MAGNIFICAT:}}	Our Savior came to be baptized, so that through the cleansing waters of baptism he might restore the old man to new life, heal our sinful nature, and clothe us with unfailing holiness.\\

\noindent\small{\uppercase{PRAYER:}}	Almighty ever-living God, who, when Christ had been baptized in the River Jordan and as the Holy Spirit descended upon him, solemnly declared him your beloved Son, grant that your children by adoption, reborn of water and the Holy Spirit, may always be pleasing to you. Through our Lord JEsus Christ, your Son, who lives and reigns with you in the unity of the Holy Spirit, one God, for ever and ever.

\begin{flushleft}\normalsize{\uppercase{VIGILS\\}}\end{flushleft}
\noindent\small{\uppercase{PSALMODY}\\} 
\uppercase{Ps 28 -- Ps 65 -- Ps 45 -- OT 29 A-- OT 45 -- OT 38}\vspace{0.5em}\\

\begin{flushleft}\normalsize{\uppercase{LAUDS\\}}\end{flushleft}
\small{\uppercase{INVITATORY:}}\normalsize
\begin{center}
\textit{Come, let us worship Christ, the beloved Son in whom the Father was well pleased.\\}
\end{center}
\noindent\small{\uppercase{PSALMODY}\\}
\uppercase{Ps 8 -- Ps 62 -- Ps 96 -- OT 19 -- Ps 147}\vspace{0.5em}

\noindent\small{\uppercase {READING:}}   Is 61:1-2a  \textbf{The spirit of the Lord GOD is upon me, because the LORD has anointed me; He has sent me to bring good news to the afflicted, to bind up the brokenhearted, to proclaim liberty to the captives, release to the prisoners, to announce a year of favor from the LORD and a day of vindication by our God.\\}

\noindent\small{\uppercase{BENEDICTUS:}}	Christ is baptized, the world is made holy; he has taken away our sins. We shall be purified by water and the Holy Spirit.\\

\begin{flushleft}\normalsize{\uppercase{SEXT\\}}\end{flushleft}
\noindent\small{\uppercase{PSALMODY}\\} 
\textsc{of Sunday, Week III of Christmas\\}

\noindent\small{\uppercase{READING:}}  Is 42:1 \textbf{ Here is my servant whom I uphold, my chosen one with whom I am pleased. Upon him I have put my spirit; he shall bring forth justice to the nations.}
\begin{center}\textit{This is my servant whom I love.\\
-My chosen one in whom I delight.}\end{center}

\begin{flushleft}\normalsize{\uppercase{SECOND VESPERS\\}}\end{flushleft}

\noindent\small{\uppercase{HYMN\\}} 
\textsc{from First Vespers}\\

\noindent\small{\uppercase{PSALMODY}\\}
\uppercase{Ps 109 -- Ps 113 -- Ps 44 -- NT 6}\vspace{0.5em}\\

\noindent\small\uppercase {READING\\}
\textsc{from First Vespers\\}

\noindent\small{\uppercase{MAGNIFICAT:}}	Christ Jesus loved us, poured out his blood to wash away our sins, and made us a kingdom and priests for God our Father. To him be glory and honor for ever.\\

\begin{center}\noindent\textsc{\small{This completes Christmas Season;\\
it is followed by the first week of Ordinary Time.}}\end{center}


