\begin{center}\normalsize{SUNDAYS OF LENT}\\
\end{center}

\begin{flushleft}\normalsize{\uppercase{FIRST VESPERS\\}}\end{flushleft}
\noindent\small{\uppercase{HYMN} p. \pageref{lent:firstHymn}-\pageref{lent:lastHymn}\\}
\noindent\small{\uppercase{PSALMODY}}
\begin{center}
\begin{tabular}{ l l }
1st Week & Week I\\
2nd Week & Week II\\
3rd Week & Week III\\
4th Week & Week IV\\
5th Week & Week I\\
\end{tabular}
\end{center}

\noindent\small{\uppercase{READING:}}
\begin{description}[labelindent=\parindent, leftmargin=*]
\item [Weeks 1-4:]    2 Cor 6:1-4a    \textbf{Working together, then, we appeal to you not to receive the grace of God in vain. For he says: “In an acceptable time I heard you, and on the day of salvation I helped you.” Behold, now is a very acceptable time; behold, now is the day of salvation. We cause no one to stumble in anything, in order that no fault may be found with our ministry; on the contrary, in everything we commend ourselves as ministers of God.\\}
\item [Week 5:]    1 Pet 1:18-21    \textbf{Realizing that you were ransomed from your futile conduct, handed on by your ancestors, not with perishable things like silver or gold but with the precious blood of Christ as of a spotless unblemished lamb. He was known before the foundation of the world but revealed in the final time for you, who through him believe in God who raised him from the dead and gave him glory, so that your faith and hope are in God.}
\end{description}

\noindent MAGNIFICAT:
\begin{description}[labelindent=\parindent, leftmargin=*]
\item [Week 1:]	Man cannot live on bread alone but by every word that comes from the mouth of God.
\item [Week 2:]	A voice spoke from the cloud: This is my beloved Son in whom I am well pleased; listen to him.
\item [Week 3:]	Now that we have been justified by faith, let us be at peace with God through our Lord Jesus Christ.
\item [Week 4:]	God loved the world so much that he gave his only Son to save all who have faith in him and to give them eternal life.
\item [Week 5:]	Unless a grain of wheat falls into the ground and dies, it remains only a single grain; but if it dies, it produces a rich harvest.
\end{description}

\noindent PRAYER:
\begin{description}[labelindent=\parindent, leftmargin=*]
\item [Week 1:]	Grant, almighty God, through the yearly observances of holy Lent, that we may grow in understanding of the riches hidden in Christ and by worthy conduct pursue their effects. Through our Lord Jesus Christ, your Son, who lives and reigns with you in the unity of the Holy Spirit, one God, for ever and ever.
\item [Week 2:]	O God, who have commanded us to listen to your beloved Son, be pleased, we pray, to nourish us inwardly by your Word, that, with spiritual sight made pure, we may rejoice to behold your glory. Through our Lord Jesus Christ, your Son, who lives and reigns with you in the unity of the Holy Spirit, one God, for ever and ever.
\item [Week 3:]	O God, author of every mercy and of all goodness, who in fasting, prayer and almsgiving have shown us a remedy for sin, look graciously on this confession of our lowliness, that we, who are bowed down by our conscience, may always be lifted up by your mercy. Through our Lord Jesus Christ, your Son, who lives and reigns with you in the unity of the Holy Spirit, one God, for ever and ever.
\item [Week 4:]	O God, who through your Word reconcile the human race to yourself in a wonderful way, grant, we pray, that with prompt devotion and eager faith the Christian people may hasten toward the solemn celebrations to come. Through our Lord Jesus Christ, your Son, who lives and reigns with you in the unity of the Holy Spirit, one God, for ever and ever.
\item [Week 5:]	By your help, we beseech you, Lord our God, may we walk eagerly in that same charity with which, out of love for the world, your Son handed himself over to death. Through our Lord Jesus Christ, your Son, who lives and reigns with you in the unity of the Holy Spirit, one God, for ever and ever.
\end{description}

\begin{flushleft}\normalsize{\uppercase{VIGILS\\}}\end{flushleft}
\noindent\small{\uppercase{PSALMODY}\\}
\uppercase{OT 34 -- OT 38 -- OT 37}\vspace{0.5em}

\begin{flushleft}\normalsize{\uppercase{LAUDS\\}}\end{flushleft}
\noindent READING:
\begin{description}[labelindent=\parindent, leftmargin=*]
\item [Weeks 1-4:]    Neh 8:9, 10   \textbf{Then Nehemiah, that is, the governor, and Ezra the priest-scribe, and the Levites who were instructing the people said to all the people: “Today is holy to the LORD your God. Do not lament, do not weep!”—for all the people were weeping as they heard the words of the law. He continued: “Go, eat rich foods and drink sweet drinks, and allot portions to those who had nothing prepared; for today is holy to our LORD. Do not be saddened this day, for rejoicing in the LORD is your strength!”\\} 
\item [Week 5:]  Lev 23:4-7  
\end{description}

\noindent BENEDICTUS
\begin{description}[labelindent=\parindent, leftmargin=*]
\item [Week 1:]	Jesus was led by the Spirit into the desert to be tempted by the devil; and when he had fasted for forty days and forty nights, he was hungry.
\item [Week 2:]	Our Lord Jesus Christ abolished death, and through the Gospel he revealed eternal life.
\item [Week 3:]	Destroy this temple, says the Lord, and in three days I will rebuild it. He was speaking of the temple of his body.
\item [Week 4:]	It was unheard of for anyone to open the eyes of a man born blind until the coming of Christ, the Son of God.
\item [Week 5:]	Our friend Lazarus has fallen asleep; let us go and wake him.
\end{description}

\begin{flushleft}\normalsize{\uppercase{SEXT\\}}\end{flushleft}
\noindent READING:
\begin{description}[labelindent=\parindent, leftmargin=*]
\item [Week 1-4:]  Is 30:15, 18  \textbf{For thus said the Lord GOD, the Holy One of Israel: By waiting and by calm you shall be saved, in quiet and in trust shall be your strength. But this you did not will. Truly, the LORD is waiting to be gracious to you, truly, he shall rise to show you mercy; for the LORD is a God of justice: happy are all who wait for him!\\}

\item [Week 5:]  1 Pet 4:13-14  \textbf{But rejoice to the extent that you share in the sufferings of Christ, so that when his glory is revealed you may also rejoice exultantly. If you are insulted for the name of Christ, blessed are you, for the Spirit of glory and of God rests upon you.}
\end{description}

\begin{center}
\textit{Turn your face away from my sins.\\
-Blot out all my guilt.}\end{center}

\begin{flushleft}\normalsize{\uppercase{SECOND VESPERS\\}}\end{flushleft}
\noindent\small{\uppercase{HYMN} p. \pageref{lent:firstHymn}-\pageref{lent:lastHymn}\\}
\noindent READING:
\begin{description}[labelindent=\parindent, leftmargin=*]
\item [Week 1-4:]  1 Cor 9:24-25  \textbf{Do you not know that the runners in the stadium all run in the race, but only one wins the prize? Run so as to win. Every athlete exercises discipline in every way. They do it to win a perishable crown, but we an imperishable one.\\}
\item [Week 5:]  Acts 13:26-30   \textbf{My brothers, children of the family of Abraham, and those others among you who are God-fearing, to us this word of salvation has been sent. The inhabitants of Jerusalem and their leaders failed to recognize him, and by condemning him they fulfilled the oracles of the prophets that are read sabbath after sabbath. For even though they found no grounds for a death sentence, they asked Pilate to have him put to death, and when they had accomplished all that was written about him, they took him down from the tree and placed him in a tomb. But God raised him from the dead.}
\end{description}

\noindent MAGNIFICAT:
\begin{description}[labelindent=\parindent, leftmargin=*]
\item [Week 1:]	Watch over us, eternal Savior; do not let the cunning tempter seize us. We place all our trust in your unfailing help.
\item [Week 2:]	Tell no one about the vision you have seen until the Son of Man has risen from the dead.
\item [Week 3:]	Whoever drinks the water that I shall give will never be thirsty again, says the Lord.
\item [Week 4:]	My son, you have been with me all the time and everything I have is yours. But we had to feast and rejoice, because your brother was dead and has come to life again; he was lost to us and now has been found.
\item [Week 5:]	When I am lifted up from the earth, I will draw all people to myself.
\end{description}